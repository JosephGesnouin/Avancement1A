\clearpage
\section{Introduction}
\label{sec:Intro}

\subsection{Contexte}
Cette thèse est réalisée au sein de l’unité “\textit{PELOPS}” (\textit{PErception LOcalization and Planning Systems for automated driving}) de l'institut Vedecom. Le laboratoire partenaire pour cette thèse est le Centre de Robotique de l'école Mines ParisTech. Le sujet d’étude concerne le lien entre la posture et l’attitude de marche afin d’en inférer des comportements. L’approche envisagée considère de l’apprentissage profond pour venir lier la position des articulations d’un piéton à une intention. L’objectif à l’issue de ces travaux est d’avoir une caractérisation de la trajectoire à venir du piéton afin d’aider à la prise de décision côté véhicule autonome.

\subsection{Sujet de thèse}
Les accidents de la route sont l'une des causes de décès les plus fréquentes au monde. Chaque année, 1,35 million de personnes, dont 270 000 piétons \cite{WinNT}, succombent des suites d'un accident routier.
La sécurité représente donc un aspect important de la circulation urbaine. C'est pourquoi des systèmes d'assistance automatisés améliorant la sécurité de ces usagers vulnérables doivent être développés.\\

Nous étudions les méthodes basées sur les caméras pour prévoir les actions futures
et les intentions des piétons dans un environnement de circulation urbaine.












