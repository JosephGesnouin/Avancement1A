\clearpage
\section{Introduction}
\label{sec:Intro}

\subsection{Contexte}
Cette thèse est réalisée au sein de l’unité “\textit{PELOPS}” (\textit{PErception LOcalization and Planning Systems for automated driving}) de l'institut Vedecom. Le laboratoire partenaire pour cette thèse est le Centre de Robotique de l'école Mines ParisTech. Le sujet d’étude concerne le lien entre la posture et l’attitude de marche afin d’en inférer des comportements. L’approche envisagée considère de l’apprentissage profond pour venir lier la position des articulations d’un piéton à une intention. L’objectif à l’issue de ces travaux est d’avoir une caractérisation de la trajectoire à venir du piéton afin d’aider à la prise de décision côté véhicule autonome.

\subsection{Sujet de thèse}
Les accidents de la route sont l'une des causes de décès les plus fréquentes au monde. Chaque année, 1,35 million de personnes, dont 270 000 piétons \cite{WinNT}, succombent des suites d'un accident routier.

Le véhicule autonome est un enjeu majeur de la mobilité de demain. Des avancées sont réalisées tous les jours pour parvenir à sa réalisation ; il reste cependant de nombreux problèmes à résoudre pour parvenir à un résultat sûr vis-à-vis des utilisateurs de la route les plus vulnérables, et notamment les piétons.
En effet, détecter et comprendre le comportement d’un être humain du point de vue du véhicule autonome est essentiel pour que celui-ci puisse prendre les bonnes décisions.

La résurgence des réseaux de neurones depuis une dizaine d’année due à l’explosion de la capacité de calculs fournit aujourd’hui de nouvelles solutions pour aborder certains problèmes impossibles à résoudre par des approches classiques. Parmi ces solutions se trouvent des classifieurs, des estimateurs, etc. qui utilisent des entrées complexes (images, vidéos, nuages de points) et qui sont capables de prédire de façon satisfaisante la nature de l’objet ou des indicateurs impossibles à définir autrement (du moins avec la même efficacité en terme de temps de calculs).

Nous étudions les méthodes basées sur les caméras pour prévoir les actions futures
et les intentions des piétons dans un environnement de circulation urbaine.












