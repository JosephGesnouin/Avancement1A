\clearpage
\chapter{Introduction}
%\section{Introduction}
\label{sec:Intro}

\section{Contexte}
Cette thèse est réalisée au sein de l’unité “\textit{PELOPS}” (\textit{PErception LOcalization and Planning Systems for automated driving}) de l'institut Vedecom. Le laboratoire partenaire pour cette thèse est le Centre de Robotique de l'école Mines ParisTech. Le sujet d’étude concerne le lien entre la posture et l’attitude de marche afin d’en inférer des comportements. L’approche envisagée considère de l’apprentissage profond pour venir lier la position des articulations d’un piéton à une intention. L’objectif à l’issue de ces travaux est d’avoir une caractérisation de la trajectoire à venir du piéton afin d’aider à la prise de décision côté véhicule autonome.

\section{Sujet de thèse}
Dans le cadre des activités en lien avec le véhicule autonome, l’Institut VEDECOM vise à développer des méthodes permettant d’anticiper le comportement de piétons dans des environnements urbains à partir de données issues de capteurs de type caméra embarqués dans le véhicule. L’objectif, à terme, est de pouvoir mieux prédire le comportement à venir dans les prochaines secondes des piétons et, en particulier, de savoir le plus tôt possible quand un piéton va décider de traverser devant le véhicule autonome et avec quelle dynamique afin de prévoir la réaction la plus appropriée.\\

L'objectif est de définir une solution exploitant l’information caméra (domaine image) et reposant sur les réseaux de neurones pour concevoir un système capable de comprendre l’intention d’un piéton en fonction de sa gestuelle (dynamique du squelette), d'indicateurs contextuels (téléphone portable, ...) et de son environnement (route / trottoir, ...) puis d’en inférer la localisation future du piéton de manière à déterminer s’il est susceptible de représenter un obstacle pour le véhicule autonome ou non.

Nous définissons l'intention comme une combinaison de comportements discrets de haut niveau ainsi que de trajectoires continues décrivant le mouvement futur attendu du piéton. L’approche considérée serait donc capable de prédire les intentions des piétons sous forme discrète et continue : 

\begin{itemize}
    \item \textbf{Actions de haut niveau} : La prédiction de l'intention discrète peut être considérée comme une classification multi-classes : \textit{Crossing / Not Crossing / About to Cross / Waiting ...}
    \item \textbf{Régression de la trajectoire} : Pour chaque piéton, nous considerons sa trajectoire comme étant une séquence de bounding box, comprenant alors les emplacements actuels et futurs du dit piéton.
\end{itemize}

Les deux facteurs principaux envisagés pour réaliser ce travail sont les suivants :

\begin{itemize}
    \item  \textbf{Le piéton} : étant donné que les informations sur le positionnement de son mouvement, sa disposition physique et l'orientation de certains membres déterminent ce qu'un conducteur utilise communément pour en déduire sa position future.
    \item \textbf{L’environnement} : L’environnement joue un rôle prépondérant lors de la prise de décision du piéton (Conditions contextuelles : nombres de voies, heure, météo, passage piéton… Situation démographique : genre, âge…)
\end{itemize}

La première question à laquelle nous souhaitons répondre étant : «Peut-on lier la dynamique des articulations d'un piéton à une intention ? », la seconde question à laquelle nous souhaitons répondre étant: "L'ajout d'informations annexes telles que l'environnement peuvent-elles améliorer la prédiction d'intention?"













