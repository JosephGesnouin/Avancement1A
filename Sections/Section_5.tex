\clearpage
\chapter{Démarche proposée au cours de la thèse}
\label{sec:SOTA}

\section{Bases de données}
Joint Attention for Autonomous Driving(JAAD) Dataset \cite{2016arXiv160904741K,Rasouli_2017_ICCV}

Pedestrian Intention Estimation (PIE) Dataset \cite{Rasouli2019PIE}

\section{Idées}

A partir de l'information brute (squelette), construire une représentation de l'information permettant une justification du conditionnement du réseaux (taille des kernels) et une explicabilité des réltats.

A partir des travaux de Guillaume Devineau et des recherches r les auto-encodeurs, proposer un outil permettant d'enrichir les bases de données existantes r la dynamique de squelettes humains


Construire une architecture de traitement via réseaux de neurones prenant en charge:
- des dynamiques à périodes longues[ 3-4s ] marche, course, ...
- des dynamiques à périodes courtes[ 1s, ] inattention (objet qui tombe)
- des meta-informations pplémentaires en entrées

Application première avec "squelette main" et Leap Motion