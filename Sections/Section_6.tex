\clearpage
\chapter{Conclusion}
\label{sec:SOTA}

Au cours de ces six mois de thèse, un état de l'art a été effectué pour plusieurs domaines de recherche dans l'optique de les combiner sous la forme de briques indépendantes afin de répondre à la problématique actuelle du sujet de thèse: «Peut-on lier la dynamique des articulations d'un piéton à une intention ? »\\

L'estimation de pose, nécessaire étape d'une prédiction dont l'analyse de la posture est un composant essentiel.\\
La reconnaissance d'actions squelettiques, afin d'obtenir une idée générale des types d'approches utilisés dans l'état de l'art pour ce format de donnée.\\
La prédiction d'intention, laissant apparaitre d'autres problématiques comparé à la reconnaissance d'actions seule: la nécéssité de réaliser une inférence sur une séquence incomplète.\\

Pour la reconnaissance d'actions et prédiction, il est intéréssant de constater que la plupart des approches complexifient et alourdissent leur modèle afin d'obtenir de meilleurs résultats. Cela nous motive à axer notre recherche sur la question de la réprésentation des données, et ce afin de minimiser la taille du réseau et par conséquent, son temps d'inférence.\\

Nous avons également présenté les réalisation accomplies au cours de cette première année de thèse portant sur cette question de la représentation. La prochaine étape sera de mettre en place une pipeline et des jeux de données annotés pour les squelettes en se basant sur les annotations des jeux de données JAAD et PIE. Ceci afin de translater les approches présentées sur des jeux de données adaptés au sujet afin de les évaluer.