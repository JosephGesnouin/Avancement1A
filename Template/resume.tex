
\vspace*{2.5cm}
\noindent\rule[2pt]{\textwidth}{0.5pt}\\
{\textbf{Résumé ---}}
Dans le cadre d’un pivot stratégique d’une entreprise de télécommunication à une entreprise numérique, l’analyse des données récupérées sur les équipements de l’univers résidentiel client permet l’analyse et l’amélioration de la qualité des services Orange. Directement rattaché à l’entité Groupe, le département DAQE (Data Analytics and Qs Enablers) fournit aux domaines d'activité stratégique des moyens de collecte et d’analyse des données des équipements maison. Durant mon apprentissage au sein de ce service, j’ai eu l’opportunité de comprendre l’ensemble de la chaîne de collecte et d’analyse des données à l'aide de technologies Big Data Orange, de mettre en pratique mes savoirs et techniques pour comprendre un environnement complexe et y développer de nouvelles briques de calcul. Pour finir, de nombreuses missions de veilles technologiques et de formations m’ont permis d’apprendre de nouvelles notions et des compétences de communication et d’organisation nécessaires au métier de data scientist.


{\textbf{Mots clés :}}
Data Mining, Chaine de collecte, Big Data, Analyse factorielle, Réseaux bayesiens, Inférence Causale, Apprentissage supervisé, Asymétrie, Livebox, Décodeur TV, Satisfaction Client, QoS.
\\
\noindent\rule[2pt]{\textwidth}{0.5pt}
\begin{center}
  Centre Universitaire des Saints-Pères, Université Paris Descartes \\
  45 Rue des Saints-Pères\\
  75006 Paris
\end{center}

